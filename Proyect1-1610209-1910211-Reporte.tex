%Plantilla HCCB para Ubuntu

\documentclass[titlepage]{article}

%Los paquetes

\usepackage{amsmath, amsfonts} %Paquetes para manejar cosas de matemática
\usepackage[spanish, es-tabla]{babel} %Paquete para utilizar LaTeX en español. La parte "es-tabla" es para que salga "TABLA 1" (en lugar de "CUADRO 1" que es el defecto)
\usepackage{graphicx} %Paquete para insertar imágenes
\usepackage{setspace} %Paquete para trabajar con las medidas en el documento
\usepackage[a4paper]{geometry} %Paquete para trabajar con el tamaño del papel y los márgenes
%\usepackage{tikz,xcolor,pgfplots,}%Paquete para hacer el gráfico
%\usepackage[dvipsnames]{xcolor}%paquete de colores
%Papel y márgenes

\geometry{top=2cm, bottom=2cm, left=2cm, right=2cm} %Márgenes
\setstretch{1} %Interlineado
\begin{document}
	\begin{titlepage}
		\centering
		\includegraphics*[scale=0.2]{logo usb.jpg}
		
		\vspace{0.5cm}		
		
		\large 
		Universidad Sim\'on Bol\'ivar\\
		Departamento de Computaci\'on y Tecnolog\'ia de la Informaci\'on\\
		Laboratorio de Algoritmos y Estructuras II - CI2692
		
		\vspace{7cm}
		
		\LARGE
		\textbf{Un algoritmo divide-and-conquer para
		el problema del agente viajero}
		
		\vspace{0.5cm}
		
		\textbf{Proyecto 1}
		
		\vspace{5cm}
		
		\large
		
		Profesor: Guillermo Palma\\
		Estudiantes:\\
		Hayde\'e Castillo Borgo. Carnet: 16-10209\\
		Jes\'us Prieto. Carnet: 19-10211
		
		\vspace{5cm}
		
		Trimestre Abril-Julio 2023
	\end{titlepage}
	
	\section{Introducci\'on}
	
	El problema del agente viajero, TSP por sus siglas en ingl\'es, es un problema de optimizaci\'on ampliamente estudiado en matem\'aticas y en computaci\'on. En este problema se tiene un conjunto de ciudades y se busca diseñar un tour que permita recorrer todas las ciudades sin repetici\'on y en la menor distancia posible.\\
	
	En el presente proyecto se considera el problema del TSP Euleriano, que es aquel en el cual las ciudades vienen dadas como puntos del plano cartesiano; y se implementa un algoritmo divide-and-conquer en conjunto con un algoritmo de optimizaci\'on 2-opt, para resolver dicho problema y presentar su soluci\'on de acuerdo al esquema de TSPLIB.\\
	
	A continuaci\'on se describen los algoritmos implementados, los resultados obtenidos y se presentan las conclusiones alcanzadas en la realizaci\'on del proyecto.
	
	\section{Diseño de la soluci\'on}
	
	Para implementar el algoritmo de divide-and-conquer se emplea el lenguaje de programaci\'on Kotlin y se organiza el c\'odigo de acuerdo a 4 algoritmos principales:
	\begin{itemize}
		\item Algoritmo 1:  divideAndConquerTSP, el cual representa el centro del esquema divide-and-conquer empleado para resolver el problema TSP
		\item Algoritmo 2:  obtenerParticiones, el cual se encarga de dividir el problema en instancias m\'as pequeñas para poder resolverlo
		\item Algoritmo 3: combinarCiclos, el cual se encarga de unir las soluciones dadas para las instancias pequeñas en una solución completa para una instancia más grande
		\item Algoritmo 4:  divideAndConquerAndLocalSearchTSP, el cual se encarga de modificar la soluci\'on proporcionada por el algoritmo divideAndConquerTSP a trav\'es del algoritmo de optimizaci\'on 2-opt y propocionar una soluci\'on m\'as eficiente al problema del TSP.
	\end{itemize}
	
	Adem\'as se tienen el Algoritmo 0, denominado obtenerDatosTSP, el cual se encarga de registrar y extraer los datos del problema TSP proporcionados a trav\'es de un archivo con formato TSPLIB; y el Algoritmo 5, denominado generarArchivoSolucionTSPLIB, el cual se encarga de procesar la soluci\'on proporcionada por los algoritmos 1-4 y generar un archivo en formato TSPLIB.\\  
	
	En esta implementaci\'on, el conjunto de ciudades se denota $P$ y viene dado por un arreglo de tripletas que contienen la coordenada $x$ (como un Double), la coordenada $y$ (como un Double) y el n\'umero respectivo de la ciudad (como un Int). Por otro lado, los tours vienen representados como ciclos que contienen los lados de las ciudades y se organizan en arreglos de pares de dichas tripletas, pues cada lado se representa como un par $(a,b)$ donde $a$ es la tripleta que representa a una ciudad y $b$ es la tripleta que representa a otra.\\
	
	En el algoritmo divideAndConquerTSP se consideran cuatro casos de acuerdo a la cantidad $n$ de ciudades en el arreglo $P$. Si $n = 1$, $n = 2$ o $n = 3$, se emplean las funciones cicloUnaCiudad, cicloDosCiudades o cicloTresCiudaes, respectivamente, para obtener un tour que recorra las $n$ ciudades en orden. Por otro lado, si $n > 4$ se procede a emplear los algoritmos obtenerParticiones y combinarCiclos para dividir y conquistar el problema.\\
	
	En el algoritmo obtenerParticiones se emplean rect\'angulos, dados por arreglos de pares de coordenadas (pares de Doubles), para poder separar el conjunto $P$ de ciudades en dos particiones m\'as pequeñas que permitan resolver el problema.
	\begin{enumerate}
		\item Primero se emplea la funci\'on obtenerRect\'angulo para obtener 4 coordenadas (pares de doubles) que delimitan un rect\'angulo que contiene a todas las ciudades de $P$. Dichas coordenadas se encuentran en el siguiente orden: esquina inferior izquierda, esquina inferior derecha, esquina superior derecha, esquina superior izquierda.
		\item  Posteriormente se determina un eje de corte de acuerdo a las dimensiones del primer rect\'angulo y este se emplea para dividir dicho rect\'angulo en dos partes, trazando una recta perpendicular a su lado m\'as largo. Dicha recta se traza en un punto determinado por la funci\'on obtenerPuntoDeCorte, la cual ordena $P$ de acuerdo a las coordenadas del eje de corte y empleando una modificaci\'on del algoritmo randomized quicksort, para luego seleccionar su punto medio como punto de corte.
		\item Luego de trazar dicha recta por medio de la funci\'on aplicarCorte, se obtienen dos rect\'angulos (izquierdo y derecho o inferior y superior) los cuales vienen dados por arreglos de pares con sus 4 coordenadas (en el mismo orden que el primer rect\'angulo) y una coordenada adicional que indica cu\'al rect\'angulo es (el izquierdo o el derecho) y respecto a cu\'al eje de corte fue obtenido. En esta \'ultima coordenada se emplea 1.0 y 0.0 para referirse al primer y al segundo rect\'angulo, respectivamente; y 2.0 y 3.0 para referirse a los ejes $X$ y $Y$, respectivamente. 
		\item Posteriormente se emplea la funci\'on obtenerPuntosRectangulo para separar $P$ en dos particiones (arreglos de tripletas) dadas por los rect\'angulo obtenidos. En este caso los puntos del lado com\'un a ambos rect\'angulos (lado de corte) se almacenan en el primero (izquierdo o inferior, seg\'un corresponda).
		\item En caso de que alguna de las particiones quedara vac\'ia, se repite el proceso anterior tomando como eje de corte el opuesto al primero. Y si, por segunda vez, alguna partici\'on queda vac\'ia, se realiza el corte por el punto medio de uno de los ejes del rect\'angulo original y se repite lo anterior. 
	\end{enumerate}
	
	En el algoritmo combinarCiclos se reciben dos ciclos obtenidos del divideAndConquerTSP y se mezclan en un s\'olo ciclo considerando la forma m\'as \'optima de unir los lados para tener la menor distancia posible. 
	
	\begin{enumerate}
		\item Primero se verifica si alguno de los ciclos recibidos es vac\'io, en cuyo caso retorna el otro ciclo.
		\item Posteriormente se verifica lado por lado en ambos ciclos para determinar cu\'al es la combinaci\'on \'optima para unirlos. Con esto se señala qu\'e lados deben removerse y qu\'e lados agregarse al momento de la uni\'on. 
		\item Una vez determinado cuales lados deben retirarse y cuales añadirse, se procede a remover los lados de los dos ciclos a trav\'es de la funci\'on \textit{remover}, y posteriormente se combinan ambos ciclos con la funci\'on \textit{tour}, ordenando los lados a agregar de tal manera que se siga la secuencia de un ciclo.
	\end{enumerate}
	
	Finalmente, en el algoritmo divideAndConquerAndLocalSearchTSP se emplea el procedimiento de optimizaci\'on 2-opt para mejorar el resultado proporcionado por divideAndConquerTSP. Dicha optimizaci\'on consiste en intercambiar los lados de un tour hasta obtener una distancia m\'as \'optima y para implementarla se emplea la funci\'on swap2OPT, la cual lleva a cabo el cambio e invierte la secci\'on correspondiente en el tour para mantener el recorrido ordenado. 
	
	\section{Resultados experimentales}
	
	A continuaci\'on se presentan los resultados de las comparaciones entre el algoritmo implementado y los resultados presentados en TSPLIB.\\
	
	El estudio experimental se llev\'o a cabo en un computador Intel® Core™ i5-2450M CPU @ 2.50GHz × 4, con 8Gb de RAM y sistema operativo Ubuntu 20.04.6 LTS. Adem\'as, se emple\'o el lenguaje de programaci\'on Kotlin en su versi\'on 1.8.21 y Java Virtual Machine JVM, versi\'on 11.0.19.\\
	
	$$\begin{array}{|c|c|c|}
		\hline
		\textbf{Nombre de la instancia} & \textbf{Distancia obtenida} & \textbf{\% Desviaci\'on del valor \'optimo}\\
		\hline
	\end{array}$$\:
	
	$$\begin{array}{|c|c|c|}
		\hline
		\textbf{Nombre del resolvedor} & \textbf{Distancia total obtenida} & \textbf{\% Desviaci\'on del valor \'optimo total}\\
		\hline
		\text{Divide-and-conquer} & & \\
		\hline
		\text{Divide-and-conquer + 2opt} & & \\
		\hline
	\end{array}$$\:
	\section{Conclusiones}
	
	Es importante destacar que la realizaci\'on del presente proyecto y el estudio del problema del viaje permitieron detallar y comprender la utilidad de los algoritmos de recursi\'on y divide-and-conquer al momento de trabajar con grandes cantidades de datos. Adem\'as, se apreci\'o como la t\'ecnica de subdividir una instancia en problemas m\'as pequeños proporciona un esquema m\'as c\'omodo y \'util para implementar algoritmos. 
\end{document}
